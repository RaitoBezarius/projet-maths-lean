\subsection{Addition World}

\textit{Addition World} est le premier monde de \textbf{Natural Number Game}. Dans ce monde, on dispose principalement de 3 tactiques: 

\textit{refl}, \textit{rw} (dont l'application était initiée dans \textit{Tutorial}) et \textit{induction}.\\
En plus, chaque théorème, une fois démontré, sera utilisé comme un résultat acquis dans les démonstrations de tous les théorèmes qui suivent. Par exemple, en commençant \textit{Addition World}, on peut utiliser les deux théorèmes suivants: add\_zero et add\_succ, qui sont supposés démontrés dans la partie \textit{Tutorial}.\\

\textit{Addition World} contient 6 niveaux: zero\_add, add\_assoc, succ\_add, add\_comm, succ\_eq\_add\_one et add\_right\_comm.
Détaillons la démonstration du théorème suivant:\\

\textbf{Le 5$^{ème}$ niveau} : succ\_eq\_add\_one \begin{center}  pour tout entier naturel $n$,  $succ(n)=n+1$ \end{center}
Preuve
\begin{minted}[mathescape,linenos,numbersep=5pt,frame=lines,framesep=2mm,breaklines]{lean}


\textbf{rw one\_eq\_succ\_zero,} : c'est plus facile de manipuler le chiffre 0 que le chiffre 1. On réécrit donc 1 en succ(0), puisque $1=succ(0)$ ( la preuve de cette égalité est one\_eq\_succ\_zero). On obtient $succ(n)=n+succ(0)$\\
 \textbf{rw add\_succ,} : add\_succ fournit l'égalité $n+succ(0)=succ(n+0)$, on l'utilise alors pour réécrire $succ(n)=n+succ(0)$ en $succ(n)=succ(n+0)$. Ainsi, on pourra utiliser un des théorèmes qui manipulent le chiffre 0\\
\textbf{rw add\_zero,} : utilisation de ce théorème pour réécrire $n+0$ en $n$\\
\textbf{refl,}
\subsection{Multiplication World}
Dans ce monde, les théorèmes reposent principalement sur les propriétés basiques de la multiplication, tels que la commutativité, l'associativité, et la distributivité de la multiplication par rapport à l'addition dans les deux sens (à gauche et à droite). \textit{Multiplication World} contient 9 niveaux: zero\_mul, mul\_one, one\_mul, mul\_add, mul\_assoc, succ\_mul, add\_mul, mul\_comm et mul\_left\_comm.\\ Nous explicitons la démonstration du théorème suivant: \\
\textbf{Le 4$^{ème}$ niveau} : mul\_add  \begin{center} La multiplication est distributive, c'est à dire pour tous entiers naturels a, b et t : $$t*(a+b)=t*a+t*b$$ \end{center}
Preuve
\textbf{induction a with d hd,} : Dans l'induction, \textbf{$a$} est renommé en \textbf{$d$} qui varie inductivement et \textbf{$hd$} est l'hypothèse d'induction sur $d$ (cas de base: $d=0$, cas d'induction: on suppose $hd$, on démontre $h(succ(d))$)  \\
\textit{\underline{Cas de base}: montrons que $t * (0 + b) = t * 0 + t * b$}\\
\textbf{rw zero\_add,} : on remplace $0+b$ par $b$, on obtient $t*b=t*0+t*b$ \\
\textbf{rw mul\_zero,} : on remplace $t*0$ par $0$, on obtient $t*b=0+t*b$ \\
\textbf{rw zero\_add,} : on obtient $t*b=t*b$ \\
\textbf{refl,} \\
\textit{\underline{Cas d'induction}: supposons $hd$ : $t*(d+b) = t * d + t * b$ et montrons $h(succ(d)):$ $t * (succ (d) + b) = t * succ (d) + t * b$ }\\
\textbf{rw succ\_add,} :une solution serait de se ramener à une équation où l'un des deux membres est égal à un membre de $hd$. Pour faire cela, on utilise succ\_add qui s'applique uniquement sur une quantité de la forme $succ(d)+b$ ($d$ et $b$ étant deux entiers naturels quelconques), nous permettant ainsi de la remplacer par $succ(d+b)$\\
\textbf{rw mul\_succ,} : on utilise \textit{mul\_succ} ($a$ $b$ : mynat) : $a * succ(b) = a * b + a $\\
\textbf{rw hd,} on remplace $t * (d + b) + t$ par $t * d + t * b+t$ en utilisant $hd$, on obtient $ t * d + t * b + t = t * succ(d) + t * b$\\
\textbf{rw add\_right\_comm,} : on applique la commutativité de l'addition pour remplacer $t * b + t$ par $ t + t * b$\\
\textbf{rw $\leftarrow$  mul\_succ, }: on utilise rw $\leftarrow$ pour remplacer $t * d + t$ (qui est le membre droit de l'égalité qui correspond au théorème mul\_succ) par $t * succ (d)$, on obtient $t * succ(d) + t * b = t * succ(d) + t * b$\\
\textbf{refl,} \\
\subsection{Power World}
Ce monde contient 8 niveaux: zero\_pow\_zero, zero\_pow\_succ, pow\_one, one\_pow, pow\_add, mul\_pow, pow\_pow et add\_squared.\\ 
Nous explicitons la démonstration du théorème suivant: \\
\textbf{Le 7$^{ème}$ niveau}: add\_squared (Cas particulier de la formule du binôme de Newton: $(a+b)^n=\sum_{k=0}^{n}{\frac{n!}{k!(n-k)!} a^nb^{n-k}}$, pour $n=2$)
\begin{center} pour tous entiers naturels $a$ et $b$ : $(a+b)^2=a^2+b^2+2*a*b $\end{center}
Preuve
 \textit{On simplifie les puissances, en réécrivant les puissances 2 en fonction de 0 }\\
  \textbf{rw two\_eq\_succ\_one,} : on utilise la preuve de $succ(1)=2$ pour réécrire le chiffre $2$ en $succ(1)$ \\
 \textbf{rw one\_eq\_succ\_zero,} :  on réécrit $1$ en $succ(0)$, on obtient donc $(a + b) ^ {succ (succ (0))} = a ^ {succ (succ (0)) }+ b ^ {succ (succ (0)) }+ succ (succ (0)) * a * b$\\
  \textbf{repeat {rw pow\_succ},} : on obtient $(a + b) ^ 0 * (a + b) * (a + b) = a ^ 0 * a * a + b ^ 0 * b * b + succ (succ (0)) * a * b$ \\
  \textbf{repeat {rw pow\_zero},} : on obtient $1 * (a + b) * (a + b) = 1 * a * a + 1 * b * b + succ (succ (0)) * a * b$\\
  \textbf{simp,} : on obtient $(a + b) * (a + b) = a * a + (b * b + a * (b * succ (succ (0))))
$, donc simp, dans ce cas, applique le théorème one\_mul(m : mynat) : $m * 1 = m$\\ 
  \textbf{repeat {rw mul\_succ},} : on obtient $(a + b) * (a + b) = a * a + (b * b + a * (b * 0 + b + b))$\\
  \textbf{simp,} : on obtient $(a + b) * (a + b) = a * a + (b * b + a * (b + b))$, donc simp, dans ce cas, applique les théorèmes mul\_zero(a : mynat):$a * 0 = 0$  et zero\_add(n : mynat):$0 + n = n$\\
  \textit{On développe (a + b) * (a + b) :} \\
 \textbf{rw mul\_add,} \\
 \textit{On développe $(a + b) * a$ :}\\
  \textbf{rw mul\_comm,} \\
  \textbf{rw mul\_add,} \\
  \textit{On développe $(a + b) * b$ :} \\
 \textbf{ rw mul\_comm (a + b) b,} \\
  \textbf{rw mul\_add,} \\
 \textbf{simp,}  \textit{On met les termes du membre de gauche dans le bon ordre } \\
  \textbf{rw $\leftarrow$ add\_assoc (a * b) (a * b) (b * b),} : on obtient $a * a + (a * b + a * b + b * b) = a * a + (b * b + a * (b + b))$\\
  \textbf{rw add\_right\_comm,} \\
  \textbf{rw add\_comm (a * b) (b * b),} \\
  \textbf{rw add\_assoc (b * b) (a * b) (a * b),} : on obtient $a * a + (b * b + (a * b + a * b)) = a * a + (b * b + a * (b + b))$\\
  \textit{On factorise par a :}\\
  \textbf{rw $\leftarrow$ mul\_add a b b,} : on obtient $a * a + (b * b + a * (b + b)) = a * a + (b * b + a * (b + b))
$\\
  \textbf{refl,}
