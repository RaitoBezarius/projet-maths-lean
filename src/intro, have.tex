\section{Tactics de base}

\paragraph{intro}

Parfois, le goal que nous cherchons � atteindre est une implication. Pour prouver que 'A -> B', on va prouver 'B sachant A vrai'. En Lean, cela revient � inclure A dans les hypoth�ses et � changer le goal en 'B'. C'est ce que fait la tactic 'intro'. On peut donner un nom � l'hypoth�se qu'on introduit : 'intro h,' ou laisser Lean choisir un nom par d�faut.
On peut �crire 'intros h1 h2 ... hn,' pour d�clarer plusieurs hypoth�ses en m�me temps.

\paragraph{have}

Pour d�clarer une nouvelle hypoth�se, on peut utiliser la tactic 'have'.
'have p : P' va s�parer le goal en 2 : montrer qu'on peut construire un �l�ment de l'ensemble P avec les hypoth�ses actuelles puis montrer le goal initial avec l'hypoth�se 'p : P' en plus.
Lorsque la preuve de l'existence de l'objet qu'on cr�e est tr�s simple, on peut contracter sa d�finition :
'have p := f a' avec 'a : A' et 'f : A -> P' comme hypoth�ses d�j� pr�sentes ajoutera directement 'p : P' dans la liste d'hypoth�ses.  