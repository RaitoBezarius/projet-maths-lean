    \subsection{\texttt{set:}}
    \texttt{set $a := t$ with $h$} est équivalent à \texttt{soit $a := t$}. Cette tactique ajoute l'hypothèse $h : a = t$ au contexte local et remplace toutes les occurrences de $t$ avec $a$.
    \subsection{\texttt{use:}}
    \texttt{use $x$} instancie le premier terme d'une existence avec $x$. On l'utilise quand le goal commence avec un $\exists$. 
    \subsection{ \texttt{obtain:}}
    Cette tactique est une combinaison de deux tactiques: \texttt{have} et \texttt{rcases}.
    \texttt{obtain \textless patt\textgreater : type}  est équivalent à \texttt{have h: type, rcases h with \textless patt\textgreater}. Si \texttt{type} n'est pas prouvé, la syntaxe à utiliser sera \texttt{obtain \textless patt\textgreater : type := proof}
    \subsection{\texttt{cases:}}
    Soit $x$ une variable dans le contexte local de type inductive, alors les hypothèses qui contiennent $x$ et le goal (si il contient $x$) se divisent selon le nombre de constructeurs inductives de $x$. Par exemple, si $x$ est un entier naturel, alors d'après l'axiome de Peano, l'induction qui permet de construire $x$ est la suivante: $d=0$ et $succ(d)=d+1$. \\ Dans ce cas, si on prend une hypothèse $h:A$ $x$ et un goal $B$ $x$ alors \texttt{cases $x$ with $d$} produit un goal $B$ $0$ avec l'hypothèse $h:A$ $0$, et un goal $B$ $succ(d)$ avec l'hypothèse $h:A$ $succ(d)$. 
    un autre exemple de variable de type inductive est donné par une hypothèse sous la forme $P\wedge Q$ ou bien $P\leftrightarrow Q$. Si on prend $h1:$ $P\wedge Q$, \texttt{cases $h1$ with $p$ $q$} va remplacer l'hypothèse $h1$ par les 2 hypothèses $p:P$, $q:Q$.
    \subsection{ \texttt{rcases:}}
     Cette tactique a le même principe que la tactique \texttt{cases}, la seule différence c'est qu'elle donne plus de flexibilité au niveau des constructeurs de l'induction correspondant à la variable à laquelle on applique \texttt{cases}. \\ Par exemple, on pose $h:(a\rightarrow b)\wedge(c\rightarrow d)$, alors \texttt{rcases h with}(\textless $A$, $B$ \textgreater \textbar \textless $C$ \textgreater) remplace h par \\ $A:a$, $B:b$, $C:c\rightarrow d$. Donc \textless $A$, $B$ \textgreater \textbar \textless $C$\textgreater divise l'hypothèse en 2 constructeurs inductives, récupère les deux premiers paramètres du premier constructeur en $A$ et $B$ et récupère le deuxième constructeur en $C$. 
