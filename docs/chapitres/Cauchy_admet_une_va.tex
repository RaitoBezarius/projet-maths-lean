\subsection{\texttt{cauchy\_admet\_une\_va}}
Ici nous allons détailler la preuve de l'unicité de la valeur d'adhérence de toute suite de Cauchy, qui, en Lean, s'énonce comme ceci : \\
\mintinline[escapeinside=||,mathescape, breaklines]{lean}{lemma cauchy_admet_une_va |$(x : \mathbb{N} \to \mathbb{R}) : \texttt{cauchy} \, x \to \forall l_1 \, l_2: X,\, \texttt{adhere}\, x\, l_1 \land \texttt{adhere}\, x\, l_2 \to l_1 = l_2$|}\\
Pour cela, on a besoin d'une autre preuve, à savoir : \\
\mintinline[escapeinside=||,mathescape, breaklines]{lean}{lemma eq_of_dist_lt |$(x\, y : X) : (\forall \varepsilon > 0,\, d\, x\, y < \varepsilon) \to x = y$| }\\
Nous l'admettrons ici (la preuve est dans le code) pour nous focaliser sur \texttt{cauchy\_admet\_une\_va}, dont nous exposerons le principe de la preuve ci- dessous. 
\begin{minted}[mathescape,linenos,numbersep=5pt,frame=lines,framesep=2mm,breaklines,escapeinside=||]{lean}
lemma cauchy_admet_une_va |$(x : \mathbb{N} \to \mathbb{R}) : \texttt{cauchy} \, x \to \forall l_1 \, l_2: X,\, \texttt{adhere}\, x\, l_1 \land \texttt{adhere}\, x\, l_2 \to l_1 = l_2$| :=
begin
  intros cauch l1 l2 h,
  apply eq_of_dist_lt,
  intros |$\varepsilon$| h|$\varepsilon$|,
  have h|$\varepsilon$|3 : |$\varepsilon/3$| > 0 := by linarith,
  obtain ⟨ n₀, h_cauchy ⟩ := cauch (|$\varepsilon/3$|) h|$\varepsilon$|3,
  obtain ⟨ p_1, ⟨ hp_1, hl₁ ⟩ ⟩ := h.1 (|$\varepsilon/3$|) h|$\varepsilon$|3 (n₀),
  obtain ⟨ p_2, ⟨ hp_2, hl₂ ⟩ ⟩ := h.2 (|$\varepsilon/3$|) h|$\varepsilon$|3 (n₀),
  have Hd := espace_metrique.triangle (x p₁) (x p₂ ) l2,
  have Hc:= h_cauchy p₁ hp₁ p₂ hp₂,
  calc
    d l1 l2 |$\leq$| d l1 (x p₁) + d (x p₁) l2 : espace_metrique.triangle _ _ _
      ... |$<$| |$\varepsilon/3$| + d (x p₁) l2 :  add_lt_add_right hl₁ (d (x p₁) l2)
      ... |$\leq$| |$\varepsilon/3$| + (d (x p₁) (x p₂) + d (x p₂) l2) : add_le_add_left Hd (|$\varepsilon/3$|)
      ... = d (x p₁) (x p₂) + (|$\varepsilon$| / 3 + d (x p₂) l2) : by ring 
      ... |$<$| |$\varepsilon/3$| + (|$\varepsilon/3$| + d (x p₂) l2) : add_lt_add_right Hc (|$\varepsilon/3$| + d (x p₂) l2)
      ... = |$\varepsilon/3$| + |$\varepsilon/3$| + d (x p₂) l2 : by ring
      ... |$<$| |$\varepsilon/3$| + |$\varepsilon/3$| + |$\varepsilon/3$| :  add_lt_add_left hl₂ (|$\varepsilon/3$| + |$\varepsilon/3$|)
      ... = |$\varepsilon$| : by ring,
end
\end{minted}
Après avoir introduit les différentes variables, on utilise \texttt{eq\_of\_dist\_lt} pour dire qu'il suffit de montrer que $\forall (\varepsilon : \mathbb{R}), \varepsilon > 0 \to d \,x\,y < \varepsilon$. On introduit les variables, utilisons la tactique $\texttt{linarith}$ qui résoud les inégalités triviales pour la propriété qui nous intéresse. \\
Ensuite, à l'aide de la tactique $\texttt{obtain}$, on dispose d'abord de $n_0$ tel que $\forall p,q \geq n_0, d\, (x\, p)\, (x\, q) < \varepsilon /3$, ce qui est possible car x est de cauchy, puis $p_1$ et $p_2$ tels que $p_1 , p_2 \geq n_0$ et $d\,(x\,p_i)\,l_i < \varepsilon /3,\, i \in \{1,2\}$, ce qui est garantit par le fait que $l_1$ et $l_2$ sont des valeurs d'adhérence de $x$.\\
Ensuite il ne reste que du calcul formel à effectuer, que l'on fait grâce à la tactique d'environnement $\texttt{calc}$, qui permet ici de partir d'un côté du résultat, et par suite d'égalités ou d'inégalités strictes ou larges - toujours par rapport à l'étape précédente - pour arriver à l'autre côté du résultat, en accolant la preuve de chaque (in)égalité à droite de celle-ci. \texttt{calc} est une tactique qui fonctionne pour n'importe quelle relation qui vérifie la transitivité. Ici on a donc utilisé l'inégalité triangulaire, la symétrie de la distance et quelques lemmes fondamentaux d'inégalité, ainsi que la tactique $\texttt{ring}$ qui résoud des égalités simples dans un anneau.\\