Ici on va démontrer que la continuité séquentielle est stable par composition, i.e. : \\
\mintinline[escapeinside=||,mathescape, breaklines]{lean}{theorem comp_seq_continue |$(f : X \to Y) \,(g : Y \to Z) :$|seq_continue|$\, f \, \land \, $|seq_continue|$ \, g \, \to \,$|seq_continue|$ \, (g \circ f)$|} \\
Tout d'abord définissons la continuité séquentielle d'une fonction, en un point puis en tous points : \\
\mintinline[escapeinside=||,mathescape, breaklines]{lean}{def seq_continue_en_l |$(f : X \to Y) \,(L : X) := \forall (x:\mathbb{N}\to X),$|converge|$ \, x \, L \to$|converge|$ \, ($|suite_image|$ \, f \, x)\, (f \, L)$|} \\ 
\\
\mintinline[escapeinside=||,mathescape, breaklines]{lean}{def seq_continue_en_l |$(f : X \to Y) \, := \forall X : L,$|seq_continue_en_l|$ \, f \, L$|}\\

Ensuite il faut définir une fonction qui renvoie la suite composée par deux fonctions, ce qu'on fait dans : \\
\mintinline[escapeinside=||,mathescape, breaklines]{lean}{theorem comp_seq_continue |$(f : X \to Y) \,(g : Y \to Z)(x : \mathbb{N} \to X) :$|suite_image|$\, g \, ($|suite_image|$ \, f \,x) = $|suite_image|$ \, (g \circ f) \, x$|} \\
On ne détaillera pas la preuve - rapide - de cette fonction.\\
On définit ensuite la continuité séquentielle en un point de la composée : 

\mintinline[escapeinside=||,mathescape, breaklines]{lean}{theorem comp_seq_continue_ponctuel |$(f : X \to Y) \,(l : X) :$|seq_continue_en_l|$\, f \, l \, \land \, $|seq_continue_en_l|$ \, g \, (f \, l) \to \,$|seq_continue_en_l|$ \, (g \circ f) \, l$|} \\

Pour cette preuve, après introduction des variables, on crée une variable, on prouve que la suite image par $g$ de la suite image par $f$ de $x$ converge vers $g(f(l))$, ce qu'on modifie grâce à la tactique $\texttt{conv}$ en « la suite image par $g \circ f$ de $x$ converge vers $g(f(l))$ », ce qui correspond exactement au but.\\

Finalement, pour prouver la continuité séquentielle en tous points de la composée, il suffit d'introduire les variables, appliquer la tactique $\texttt{apply}$ au théorème précédent, puis montrer grâce aux hypothèses que pour tout point $l:X$, $f$ est continue en $l$ et $g$ en $f(l)$.\\
