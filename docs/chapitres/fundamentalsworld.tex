\subsection{Addition World}
\textit{Addition World} est le premier monde de \textbf{Natural Number Game}. Dans ce monde, on dispose principalement de 3 tactiques: \texttt{refl}, \texttt{rw} (dont l'application était initiée dans \textit{Tutorial}) et \texttt{induction}.\\
En plus, chaque théorème, une fois démontré, sera utilisé comme un résultat acquis dans les démonstrations de tous les théorèmes qui suivent. Par exemple, en commençant \textit{Addition World}, on peut utiliser les deux théorèmes suivants: add\_zero et add\_succ, qui sont supposés démontrés dans la partie \textit{Tutorial}.
\textit{Addition World} contient 6 niveaux: \texttt{zero\_add}, \texttt{add\_assoc}, \texttt{succ\_add}, \texttt{add\_comm}, \texttt{succ\_eq\_add\_one} et \texttt{add\_right\_comm}.
Détaillons la démonstration du théorème suivant:\\
\subsubsection{Niveau 5} \texttt{succ\_eq\_add\_one} \begin{center}   $\forall n \in \mathbb{N},$  $succ(n)=n+1$ \end{center}

\begin{minted}[mathescape,linenos,numbersep=5pt,frame=lines,framesep=2mm,breaklines]{lean}
theorem succ_eq_add_one (n : mynat) : succ n = n + 1 :=
begin [nat_num_game]
rw one_eq_succ_zero, -- c'est plus facile de manipuler le chiffre 0 que le chiffre 1. 
--On réécrit donc 1 en succ(0), puisque 1=succ(0) ( la preuve de cette égalité est
--one_eq_succ_zero). On obtient succ(n)=n+succ(0)
 rw add_succ, -- add_succ fournit l'égalité n+succ(0)=succ(n+0), on l'utilise alors pour réécrire succ(n)=n+succ(0) en succ(n)=succ(n+0). Ainsi, on pourra utiliser un des théorèmes qui manipulent le chiffre 0
rw add_zero, -- utilisation de ce théorème pour réécrire n+0 en n
refl,
end
\end{minted}
\subsection{Multiplication World}
Dans ce monde, les théorèmes reposent principalement sur les propriétés basiques de la multiplication, tels que la commutativité, l'associativité, et la distributivité de la multiplication par rapport à l'addition dans les deux sens (à gauche et à droite). \textit{Multiplication World} contient 9 niveaux: \texttt{zero\_mul}, \texttt{mul\_one}, \texttt{one\_mul}, \texttt{mul\_add}, \texttt{mul\_assoc}, \texttt{succ\_mul}, \texttt{add\_mul}, \texttt{mul\_comm} et \texttt{mul\_left\_comm}.\\ Nous explicitons la démonstration du théorème suivant: \\
\subsubsection{Niveau 4}  \texttt{mul\_add}  \begin{center} La multiplication est distributive à gauche, c'est à dire  $\forall a, b, t \in \mathbb{N},$  $t\times(a+b)=t\times a+t\times b $ \end{center}
\begin{minted}[mathescape,linenos,numbersep=5pt,frame=lines,framesep=2mm,breaklines,escapeinside=||]{lean}
lemma mul_add (t a b: mynat): t*(a+b) = t*a+t*b :=
begin [nat_num_game]
induction a with d hd, -- Dans l'induction, a est renommé en d qui varie inductivement 
--et hd est l'hypothèse d'induction sur d (cas de base: d=0, cas d'induction: on suppose hd, 
-- on démontre h(succ(d)))   
--Cas de base: montrons que t$\times$(0+b) = t$\times$0+t$\times$b
rw zero_add, -- on remplace 0+b par b, on obtient t$\times$b = t$\times$0+t$\times$b 
rw mul_zero, -- on remplace t$\times$0 par 0, on obtient t$\times$b = 0+t$\times$b 
rw zero_add, -- on obtient t$\times$b = t$\times$b
refl,
--Cas d'induction: supposons hd : t$\times$(d+b) = t$\times$d + t $\times$ b
--et montrons h(succ(d)): t$\times$(succ(d)+b) = t$\times$succ(d)+t$\times$b 
rw succ_add, -- une solution serait de se ramener à une équation où l'un des deux membres 
--est égal 
--à un membre de hd. 
--Pour faire cela, on utilise succ_add qui s'applique uniquement sur une quantité 
--de la forme succ(d)+b (d et b étant deux entiers naturels quelconques), nous permettant ainsi 
--de la remplacer par succ(d+b)
rw mul_succ, -- on utilise mul_succ (a b: mynat): a$\times$succ(b) = a$\times$b+a
rw hd, -- on remplace t$\times$(d+b)+t par t$\times$d+t$\times$b+t en utilisant hd, 
-- on obtient t$\times$d+t*b+t = t$\times$succ(d)+t$\times$b
rw add_right_comm, -- on applique la commutativité de l'addition pour remplacer t$\times$b+t par t+t$\times$b
rw |$\to$|  mul_succ, --on utilise rw $\leftarrow$ pour remplacer t$\times$d +t (qui est le membre droit
-- de l'égalité qui correspond au théorème mul_succ) par t$\times$succ(d), 
-- on obtient t$\times$succ(d)+t$\times$b = t$\times$succ(d)+t$\times$b
refl
end
\end{minted}
\subsection{Power World}
Ce monde contient 8 niveaux: \texttt{zero\_pow\_zero}, \texttt{zero\_pow\_succ}, \texttt{pow\_one}, \texttt{one\_pow}, \texttt{pow\_add}, \texttt{mul\_pow}, \texttt{pow\_pow} et \texttt{add\_squared}.
Nous explicitons la démonstration du théorème suivant:\\
\subsubsection{Niveau 7} \texttt{add\_squared} (Cas particulier de la formule du binôme de Newton: $(a+b)^n=\sum_{k=0}^{n}{\frac{n!}{k!(n-k)!} a^nb^{n-k}}$, pour $n=2$)
\begin{center} $\forall a, b \in \mathbb{N}$, $(a+b)^2=a^2+b^2+2*a*b $\end{center}
\begin{minted}[mathescape,linenos,numbersep=5pt,frame=lines,framesep=2mm,breaklines,escapeinside=||]{lean}
lemma pow_pow (a m n: mynat): (a^m)^n = a^(m*n) :=
begin [nat_num_game]
 -- on simplifie les puissances, en réécrivant les puissances 2 en fonction de 0 
 rw two_eq_succ_one, -- on utilise la preuve de succ(1)=2 pour réécrire le chiffre 2 en succ(1)
 rw one_eq_succ_zero, -- on réécrit 1 en succ(0), on obtient donc 
 --$(a + b) ^ {succ (succ (0))} = a ^ {succ (succ (0)) }+ b ^ {succ (succ (0)) }+ succ (succ (0)) \times a \times b$
 repeat {rw pow_succ}, -- on obtient $(a + b) ^ 0 \times (a + b) \times (a + b) = a ^ 0 \times a \times a + b ^ 0 \times b \times b + succ (succ (0)) \times a \times b$ 
 repeat {rw pow_zero}, -- on obtient $1 \times (a + b) \times (a + b) = 1 \times a \times a + 1 \times b \times b + succ (succ (0)) \times a \times b$
 simp, -- on obtient $(a + b) \times (a + b) = a \times a + (b \times b + a \times (b \times succ (succ (0))))$
--donc simp, dans ce cas, applique le théorème one_mul (m: mynat): $m \times 1 = m$
 repeat {rw mul_succ}, -- on obtient $(a + b) * (a + b) = a * a + (b * b + a * (b * 0 + b + b))$
 simp, -- on obtient $(a + b) \times (a + b) = a \times a + (b \times b + a \times (b + b))$
 -- donc simp, dans ce cas, applique les théorèmes mul_zero (a: mynat): $a \times 0 = 0$ 
 -- et zero_add (n: mynat): $0 + n = n$
 -- on développe $(a + b) \times (a + b)$ :
 rw mul_add,
 -- on développe $(a + b) \times a$ 
 rw mul_comm,
 rw mul_add,
 -- on développe $(a + b) \times b$ 
 rw mul_comm (a + b) b,
 rw mul_add,
 simp, -- on met les termes du membre de gauche dans le bon ordre 
 rw |$\leftarrow$| add_assoc (a * b) (a * b) (b * b), -- on obtient $a \times a + (a \times b + a \times b + b \times b) = a \times a + (b \times b + a \times (b + b))$
 rw add_right_comm,
 rw add_comm (a * b) (b * b),
 rw add_assoc (b * b) (a * b) (a * b), -- on obtient 
 -- $a \times a + (b \times b + (a \times b + a \times b)) = a \times a + (b \times b + a \times (b + b))$
 -- on factorise par a :
 rw |$\leftarrow$| mul_add a b b, -- on obtient $a \times a + (b \times b + a \times (b + b)) = a \times a + (b \times b + a \times (b + b))$
 refl
end
\end{minted}
