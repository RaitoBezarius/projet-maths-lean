\subsection{\texttt{cauchy\_est\_bornee}}
\subsubsection{Définition}
Soit $(X,d)$ un espace métrique. On dit qu'une suite des éléments de $X$, $(u_n)_{n\in \mathbb{N}}$ est de Cauchy si est seulement si: $$\forall \varepsilon \in \mathbb{R_+^*}, \exists n_0 \in \mathbb{N}, \forall n,m \in \mathbb{N}, n,m \geq n_0 \Rightarrow d(U_n,U_m) \leq \varepsilon$$ 
\subsubsection{\'Enoncé du lemme}
Dans un espace métrique, toute suite de Cauchy est bornée.

\subsubsection{L'idée principale de la preuve}
On veut montrer que toute suite de Cauchy est bornée. Une suite bornée dans un espace métrique $(W_n)_{n\in\mathbb{N}}$ est une suite qui vérifie la condition suivante:$$ \forall m \in \mathbb{N}, \exists M , \forall n \in \mathbb{N}, d(W_n,W_m)\leq M $$
Soit $(U_n)_{n\in\mathbb{N}}$ une suite de Cauchy.
D'après la définition d'une suite de Cauchy, en prenant 1 comme valeur de $\varepsilon$, il existe un rang $n_0 \in \mathbb{N}$ tel que pour tout $n,m \geq n_0$, $d(U_n,U_m)\leq 1$.\\ Donc on peut diviser les termes de la suite en deux ensembles: \\ $E_1=\big\{U_n\mid n\leq n_0 \big\}$ et $E_2=\big\{U_n\mid n > n_0 \big\}$. \\
Soit $Y$ un élément quelconque de la suite $(U_n)$.
\begin{itemize}
    \item $\big\{U_n\mid n\leq n_0 \big\}$ est un ensemble fini dénombrable, donc pareil pour l'ensemble $D_1=\big\{d(U_n,Y), \forall U_n\in E_1\big\}$ qui est également dénombrable et fini.\\ Alors l'ensemble $D_1$ admet un maximum $M_1$.
    \item Soit $D_2=\big\{d(U_n,Y)\mid n> n_0 \big\}$. D'après l'inégalité triangulaire, $d(U_n,Y)\leq d(U_n,U_{n_0})+d(U_{n_0},Y)$.\\ Or d'après le caractère Cauchy de la suite, $d(U_n,U_{n_0})\leq 1$ pour tout $n\geq n_0$.\\ On en déduit que $D_2$ est majoré par $d(U_{n_0},Y)+1$. 
\end{itemize}
On en déduit que la suite $(U_n)$ est majorée par $\max(M_1,d(U_{n_0},Y)+1)$. Elle est donc bornée.\\


Dans la partie ci-dessous, nous allons expliquer l'utilisation de certaines tactiques dans la démonstration de ce théorème sur Lean.\\
\begin{itemize}
    \item \texttt{obtain $\langle N, H \rangle $}: $\exists N, \forall p \geq N, \forall q \geq N, (d$ $(x_p)$ $(x_q)) < 1$
    
    On utilise cette tactique pour stocker dans $H$ une formule mathématique (hypothèse) qui dépend d'un entier $N$, et dans $N$ l'entier en question. \\ Suite à l'utilisation de \texttt{obtain}, on doit démontrer l'existence d'un entier $N$ qui vérifie $H$.\\ Pour ce faire, on utilise l'hypothèse de Cauchy sur $x$ et le fait que 1 soit un entier strictement positif.\\
    \'A ce niveau, le \textbf{goal} est: $\exists (M:\mathbb{R}), M > 0 \wedge \forall (n : \mathbb{N})$ $d$ $(x_n)$ $y \leq M$.
    \item \texttt{set Limage:= $\big\{ M: \mathbb{R} \mid \exists n\leq N, M = d$ $(x_n)$ $y\big\}$}
    
    
    Cet ensemble contient toutes les valeurs possibles de la distance entre un terme de la suite d'indice inférieur ou égal à $N$ et $y$ (qui est un terme fixe quelconque).\\ Afin de majorer cet ensemble, on peut démontrer qu'il est non vide et fini.
    \item \texttt{have limage\_finiteness: Limage.finite}
    
    
     Cette tactique permet d'ajouter une hypothèse nommée \texttt{limage\_finiteness} qui dit que l'ensemble \texttt{Limage} est fini.\\ Elle est suivie par une démonstration dont la démarche est la suivante:
    \begin{itemize}
         \item On a défini la fonction \texttt{fonction\_distance} qui prend comme paramètres une suite $x$, un indice $n$ et un terme $y$ et retourne la distance entre $y$ et $x_n$. \\ On montre que l'ensemble \texttt{Limage} est l'image de l'ensemble $\big\{i:\mathbb{N}\mid i\leq \mathbb{N}\big\}$ par \texttt{fonction\_distance}.
        \item On utilise \texttt{apply set.finite\_image}, pour appliquer le résultat suivant: l'image d'un ensemble fini est finie. \\ Le \texttt{goal} devient de démontrer que l'ensemble $\big\{i:\mathbb{N}\mid i\leq \mathbb{N}\big\}$ est fini.
        \item On utilise \texttt{exact set.finite\_le\_nat $N$} pour ce faire.
    \end{itemize}
    \item \texttt{have limage\_nonempty: Limage.nonempty} 
    
    
    Cette tactique permet d'ajouter une hypothèse nommée \texttt{limage\_nonempty} qui dit que l'ensemble Limage est non vide.\\ Afin de démontrer ce résultat, on utilise le fait que $d$ $(x_0)$ $y  \in \textrm{Limage}$ puisque $0 \leq N$.
    \item \texttt{have sup\_est\_atteint: Sup Limage $\in$ Limage}
    
    
    Pour démontrer \texttt{sup\_est\_atteint} (c'est à dire que \texttt{Limage} admet un maximum), on utilise \texttt{set.finite.has\_a\_reached\_sup} avec les hypothèses \texttt{limage\_finiteness} et \texttt{limage\_nonempty}.\\ Donc, il existe un entier $n\leq N$ tel que $d$ $(x_n)$ $y= \max$(\textrm{Limage}).
    \item \texttt{use $(\max (d$ $(x_n)$ $y,$ $1+d$ $(x_N)$ $y))$}
    
    
    De cette façon le \texttt{goal} devient de démontrer que $\max(d$ $(x_n)$ $y ,$ $1+d$ $(x_N)$ $y)$ est un majorant de la suite de Cauchy $x$ en question. \newline C'est à dire qu'il faut montrer que $\max(d$ $(x_n)$ $y , 1+d$ $(x_N)$ $y) > 0$ (ce qui est immédiat), \\ et que $\forall n_1 : \mathbb{N}, d$ $(x_{n_1})$ $y \leq \max (d$ $(x_n)$ $y , 1+ d$ $(x_N)$ $y)$
    \item \texttt{intro p, by\_cases (p \geq N)}
    
    
    C'est à dire qu'on prend un entier $p$ et on montre la propriété dans deux cas: pour $p \geq N$ et pour $p < N$. \\
    Dans le cas où $p \geq N$: \\ D'après l'hypothèse $H$, puisque $p \geq N$ et $N \geq N$, $d$ $(x_p)$ $(x_N) < 1$. \\ Alors $d$ $(x_p)$ $(x_N) + d$ $(x_N)$ $y \leq 1 + d$ $(x_N)$ $y$. \\ Donc à fortiori $d$ $(x_p)$ $(x_N) + d$ $(x_N)$ $y \leq \max (d$ $(x_n)$ $y ,$ $1 + d$ $(x_N)$ $y)$. 
    \\ Or d'après l'inégalité triangulaire $d$ $(x_p)$ $y \leq d$ $(x_p)$ $(x_N) + d$ $(x_N)$ $y$. \\ Donc $d$ $(x_p)$ $y \leq \max (d$ $(x_n)$ $y ,$ $1 + d$ $(x_N)$ $y)$.\\
    Dans le cas où $p < N$: \newline On applique l'hypothèse \texttt{sup\_atteint} pour majorer $d$ $(x_p)$ $y$ par $d$ $(x_n)$ $y$, puisque $d$ $(x_n)$ $y$ est la borne supérieure de l'ensemble \texttt{Limage} et $d$ $(x_p)$ $y \in$ \texttt{Limage}.\\
    Donc $d$ $(x_p)$ $y \leq \max (d$ $(x_n)$ $y ,$ $1 + d$ $(x_N)$ $y)$.
\end{itemize}
