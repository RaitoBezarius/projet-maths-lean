\section{Lemma Cauchy\_est\_bornee}
\subsection{Définition}
Soit $(X,d)$ un espace métrique. On dit qu'une suite des éléments de $X$, $(u_n)_{n\in \mathbb{N}}$ est de Cauchy si est seulement si: $$\forall \varepsilon \in \mathbb{R_+^*}, \exists n_0 \in \mathbb{N}, \forall n,m \in \mathbb{N}, n,m \geq n_0 \Rightarrow d(U_n,U_m) \leq \varepsilon$$ 
\subsection{\'Enoncé du lemme}
Dans un espace métrique, toute suite de Cauchy est bornée.

\subsection{L'idée principale de la preuve}
On veut montrer que toute suite de Cauchy est bornée. Une suite bornée dans un espace métrique $(W_n)_{n\in\mathbb{N}}$ est une suite qui vérifie la condition suivante:$$ \forall m \in \mathbb{N}, \exists M , \forall n \in \mathbb{N}, d(W_n,W_m)\leq M $$
Soit $(U_n)_{n\in\mathbb{N}}$ une suite de Cauchy.
D'après la définition d'une suite de Cauchy, en prenant 1 comme valeur de $\varepsilon$, il existe un rang $n_0 \in \mathbb{N}$ tel que pour tout $n,m \geq n_0$, $d(U_n,U_m)\leq 1$. Donc on peut diviser les termes de la suite en 2 ensembles: $E_1=\big\{U_n|n\leq n_0 \big\}$ et $E_2=\big\{U_n|n > n_0 \big\}$.
Soit $Y$ un élément quelconque de la suite $(U_n)$.
\begin{itemize}
    \item $\big\{U_n|n\leq n_0 \big\}$ est un ensemble fini dénombrable, donc pareil pour l'ensemble $D_1=\big\{d(U_n,Y), \forall U_n\in E_1\big\}$ qui est également dénombrable et fini. Alors l'ensemble $D_1$ admet un maximum $M_1$.
    \item Soit $D_2=\big\{d(U_n,Y)|n> n_0 \big\}$. D'après l'inégalité triangulaire, $d(U_n,Y)\leq d(U_n,U_{n_0})+d(U_{n_0},Y)$. Or d'après le caractère Cauchy de la suite, $d(U_n,U_{n_0})\leq 1$ pour tout $n\geq n_0$, on en déduit que $D_2$ est majoré par $d(U_{n_0},Y)+1$. 
\end{itemize}
On en déduit que la suite $(U_n)$ est majorée par $max(M_1,d(U_{n_0},Y)+1)$. Elle est donc bornée.\\


Dans la partie ci-dessous, nous allons expliquer l'utilisation de certaines tactiques dans la démonstration de ce théorème sur Lean.\\
\begin{itemize}
    \item obtain \textless $N, H$ \textgreater : $\exists N, \forall p \geq N, \forall q \geq N,$ ((d (x p) (x q)) $< 1$)
    
    on utilise cette tactique pour stocker dans $H$ une formule mathématique (hypothèse) qui dépend d'un entier $N$, et dans $N$ l'entier en question. Suite à l'utilisation de \texttt{obtain}, on doit démontrer l'existence d'un entier $N$ qui vérifie $H$. Pour ce faire, on utilise l'hypothèse \texttt{cauch: cauchy x} et le fait que 1 soit un entier strictement positive.\\
    \'A ce niveau, le \textbf{goal} est: $\exists (M:\mathbb{R}), M > 0 \wedge \forall (n : \mathbb{N})$ $d$ $(x n)$ $y \leq M$
    \item set Limage:= $\big\{ M: \mathbb{R}, \exists n\leq N,$ M = d (x n) y $\big\}$
    
    
    Cet ensemble contient toutes les valeurs possibles de la distance entre un terme de la suite d'indice inférieur ou égal à $N$ et $y$ (qui est un terme fixe quelconque de la suite). On cherche à majorer cet ensemble. Une méthode pour le faire sera de démontrer qu'il est non vide et fini.
    \item have limage\_finiteness: Limage.finite
    
    
     Cette tactique permet d'ajouter une hypothèse nommée \texttt{limage\_finiteness} qui dit que l'ensemble \texttt{Limage} est fini. Elle est suivie par une démonstration dont la démarche est la suivante:
    \begin{itemize}
         \item On a défini la fonction \texttt{fonction\_distance} qui prend comme paramètres une suite $X$, un indice $n$ et un terme $Y$ de la suite et retourne la distance entre $Y$ et $X_n$. On montre que l'ensemble \texttt{Limage} est l'image de l'ensemble $\big\{i:\mathbb{N}, i\leq \mathbb{N}\big\}$ par \texttt{fonction\_distance}.
        \item On utilise \texttt{apply set.finite\_image}, pour appliquer le résultat suivant: l'image d'un ensemble fini est finie. Le \texttt{goal} devient de démontrer que l'ensemble $\big\{i:\mathbb{N}, i\leq \mathbb{N}\big\}$ est fini.
        \item On utilise \texttt{exact set.finite\_le\_nat $N$} pour ce faire.
    \end{itemize}
    \item have limage\_nonempty: Limage.nonempty 
    
    
    Cette tactique permet d'ajouter une hypothèse nommée \texttt{limage\_nonempty} qui dit que l'ensemble Limage est non vide. Afin de démontrer ce résultat, on utilise le fait que $d$ $(x 0)$ $y$ appartient à l'ensemble, puisque $0 \leq N$.
    \item have sup\_est\_atteint: Sup Limage ∈ Limage
    
    
    Pour démontrer \texttt{sup\_est\_atteint} (c'est à dire que \texttt{Limage} admet un maximum), on utilise \texttt{set.finite.has\_a\_reached\_sup} avec les hypothèses \texttt{limage\_finiteness} et \texttt{limage\_nonempty}. On note ce maximum  $d (x n) y$.
    \item use $(max (d (x n) y) (1 + d (x N) y))$
    
    
    De cette façon le \texttt{goal} devient de démontrer que $max (d (x n) y) (1 + d (x N) y)$ est un majorant de la suite de Cauchy $X$ en question. C'est à dire il faut montrer que $max (d (x n) y) (1 + d (x N) y)$ $>$ $0$ (ce qui est trivial) et que $\forall n_1 : \mathbb{N}, d (xn_1) y \leq max (d (x n) y) (1 + d (x N) y)$
    \item intro p, by\_cases (p ≥ N)
    
    
    C'est à dire qu'on prend un entier $p$ et on montre la propriété dans 2 cas: pour $p \geq N$ et pour $p < N$. \\
    Dans le cas où $p \geq N$, d'après l'hypothèse $H$, puisque $p \geq N$ et $N \geq N$, $d (x p) (x N) < 1$. \\ Alors $d (x p) (x N) + d (x N) y \leq 1 + d (x N) y$ \\ Donc à fortiori $d (x p) (x N) + d (x N) y \leq max (d (x n) y) (1 + d (x N) y)$. Or d'après l'inégalité triangulaire $d (x p) y \leq d (x p) (x N) + d (x N) y$, donc $d (x p) y \leq max (d (x n) y) (1 + d (x N) y)$\\
    Dans le cas où $p < N$, on applique l'hypothèse \texttt{sup\_atteint} pour majorer $d (x p) y$ par $d (x n) y$ puisque $d (x n) y$ est la borne supérieure de l'ensemble \texttt{Limage} et $d (x p) y$ appartient à \texttt{Limage}.\\
    Donc $d (x p) y \leq max (d (x n) y) (1 + d (x N) y)$.
\end{itemize}
